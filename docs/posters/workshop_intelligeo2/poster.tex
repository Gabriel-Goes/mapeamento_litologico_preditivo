\documentclass[12pt]{article}
%\usepackage[margin=1in, paperwidth=35.43in, paperheight=47.24in]{geometry}
\usepackage[a0paper, margin=1in]{geometry}
\usepackage{graphicx}
\usepackage{tikz}
\usepackage{lipsum} % Para gerar texto de exemplo
\usepackage{anyfontsize}
\usepackage{caption}
% Configura para porguês do Brasil
\usepackage[T1]{fontenc}
\usepackage[utf8]{inputenc}
\usepackage[portuguese]{babel}
\usepackage{parskip}
\usepackage{ragged2e}


\pagestyle{empty} % Remove número de página e cabeçalhos

%%%%%%%%%%%%%%%%%%%%%%%%%%%%%%%%%%%%%%%%%%%%%%%%%%%%%%%%%%%%%%%%%%%%%%%%%%%%%%%
\begin{document}

% Título, autores, e filiação
\begin{center}
    \textbf{
    {\fontsize{50}{60}\selectfont Da Terra ao Código: Integrando Dados Geológicos\\
                                        à Inteligencia Computacional}\\[20mm] % Tamanho do título
    {\fontsize{40}{48}\selectfont Gabriel Góes Rocha de Lima}\\[5mm] % Tamanho dos autores
    {\fontsize{35}{42}\selectfont Universidade de São Paulo}\\[20mm] % Tamanho da filiação
    \rule{1\textwidth}{0.4pt} % Linha horizontal
    }
\end{center}

% Inserir logos
\begin{tikzpicture}[remember picture,overlay]
    \node[anchor=north west, yshift=-1.5cm, xshift=2cm] at (current page.north west) 
        {\includegraphics[width=10cm]{../../resumos/logo.png}}; % Ajuste o caminho e tamanho
    \node[anchor=north east, yshift=-7.8cm, xshift=-1.8cm] at (current page.north east) 
        {\includegraphics[width=15cm]{../imgs/IGC.png}}; % Ajuste o caminho e tamanho
    \node[anchor=north east, yshift=-1.8cm, xshift=-5cm] at (current page.north east) 
        {\includegraphics[width=10cm]{../imgs/INRS.png}}; % Ajuste o caminho e tamanho
\end{tikzpicture}

% Arquivo para organizar e encapsular o conteúdo do Poster




% Linha horizontau como footer antes das informações adicionais
\begin{tikzpicture}[remember picture,overlay]
% desenhe uma linha horizontal com textwidth de comprimento e 10cm acima do sul da página
%     \draw (current page.south west) ++(2,10cm) -- ++(\textwidth,0);
    \node[anchor=center, yshift=8.5cm] at (current page.south) 
        {\noindent\rule{\textwidth}{2pt}};
\end{tikzpicture}

% Informações adicionais no rodapé
\begin{tikzpicture}[remember picture,overlay]
    \node[anchor=south west, yshift=7cm, xshift=-30cm] at (current page.south) 
        {\fontsize{20}{26}\selectfont \textbf{Referências}: Karianne J. Bergen et al. ,Machine learning for data-driven discovery in solid Earth geoscience.Science363,eaau0323(2019).DOI:10.1126/science.aau0323};
    \node[anchor=south west, yshift=4.5cm, xshift=-30cm] at (current page.south) 
        {\fontsize{40}{46}\selectfont acesse os links ou entre em cotato.};
    \node[anchor=south west, yshift=0.2cm, xshift=-39cm] at (current page.south) 
        {\includegraphics[width=8cm]{../imgs/qrcode_github.com.png}}; % Ajuste o caminho e tamanho
    \node[anchor=south west, yshift=2.5cm, xshift=-30cm] at (current page.south) 
        {\fontsize{40}{46}\selectfont correio eletrônico: \textbf{gabrielgoes@usp.br}}; % Ajuste conforme necessário
    \node[anchor=south west, yshift=1cm, xshift=-30cm] at (current page.south) 
        {\fontsize{40}{46}\selectfont \textbf{www.github.com/Gabriel-Goes/mapeamento\_litologico\_preditivo}};
    \node[anchor=south east, yshift=0cm, xshift=45cm] at (current page.south) 
        {\includegraphics[width=25cm]{../imgs/IGC.png}}; % Ajuste o caminho e tamanho
\end{tikzpicture}



\end{document}
