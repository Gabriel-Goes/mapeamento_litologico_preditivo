% Autor: Gabriel Góes Rocha de Lima
% Descrição: Resumo do postesr para o II workshop Inteligência Artificial e Geociências

\documentclass[11pt]{article} % declaração do tipo de documento
\usepackage[utf8]{inputenc} % declaração do tipo de codificação
\usepackage{geometry} % pacote para definição de margens
\usepackage{ragged2e} % Texto justificado
\usepackage{indentfirst} % pacote para indentação do primeiro parágrafo
\usepackage{setspace} % pacote para definição de espaçamento
\usepackage{titlesec} % Pacote para formatação de títulos

\geometry{a4paper, left=2cm, right=2cm, top=2cm, bottom=2cm} % definição das margens
\onehalfspacing % espaçamento de 1.5
\setlength{\parindent}{1.25cm} % indentação de 1.25cm
\setlength{\parskip}{0.5em} % set distance spacing between paragraphs
%\renewcommand{\baselinestretch}{1.5} % set line spacing
%\setmainfont{SourceCodePro Nerd Font} % Definindo a fonte principal
\titleformat{\section}{\normalfont\bfseries}{\thesection}{1em}{} % Formatação de títulos de seção
\titleformat{\subsection}{\normalfont\bfseries}{\thesubsection}{1em}{} % Formatação de títulos de subseção


\begin{document} % início do documento
\title{Da Terra ao Código: Integrando Dados Geológicos com Inteligência Computacional} % título do documento
\author{Gabriel Góes Rocha de Lima} % autor do documento
\date{28 de fevereiro de 2024} % data do documento
\maketitle % criação do título

\par{
Nosso trabalho atual representa uma ideia ambiciosa ainda em fase alfa de desenvolvimento,
visando a criação de um protótipo revolucionário para a exploração mineral. Este conceito inovador
foca na capacidade de gerar, em tempo real, mapas preditivos altamente precisos para qualquer
quadrícula geográfica específica, com a flexibilidade de incorporar continuamente novas
informações ao banco de dados, refinando assim a acurácia dos mapas gerados.
}

\par{

A fundação deste sistema em construção é a aplicação de tecnologias avançadas de
inteligência artificial (IA) e aprendizado de máquina, que analisam e processam
grandes volumes de dados geológicos e geoespaciais. Este método permite não apenas
a identificação de litologias e potenciais depósitos minerais mas também adapta-se
dinamicamente à inserção de novos dados, melhorando progressivamente a precisão
de suas classificações.
}

\par{Estamos utilizando o PostgreSQL, com a extensão PostGIS, como espinha dorsal
para gerenciar a complexidade dos dados geoespaciais. Esta escolha nos oferece uma
plataforma robusta para armazenamento e análise de dados, essencial para suportar o
processamento intensivo necessário para nosso sistema. A arquitetura foi especialmente
pensada para facilitar a expansão e a atualização contínua do banco de dados, uma
característica chave para o sucesso do projeto a longo prazo.}

\par{Uma inovação crítica em desenvolvimento é a capacidade do sistema de avaliar
a precisão de cada mapa produzido por métricas específicas, selecionando automaticamente
a versão mais acurada para armazenamento e referência futura. Este mecanismo assegura uma
melhoria contínua na qualidade dos mapas preditivos disponíveis, otimizando a
eficácia da exploração mineral.}


\end{document} % fim do documento

